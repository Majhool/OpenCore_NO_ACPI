\documentclass[]{article}

\usepackage{lmodern}
\usepackage{amssymb,amsmath}
\usepackage{ifxetex,ifluatex}
\usepackage{fixltx2e} % provides \textsubscript
\usepackage[T1]{fontenc}
\usepackage[utf8]{inputenc}
\usepackage{upquote}
\usepackage{microtype}
\usepackage[unicode=true]{hyperref}
\usepackage{longtable,booktabs}
\usepackage{footnote}
\usepackage{listings}
\usepackage{mathtools}
\usepackage{parskip}
\usepackage[margin=0.7in]{geometry}
\usepackage{titlesec}
\usepackage[yyyymmdd,hhmmss]{datetime}
\usepackage{textcomp}
\usepackage{tikz}

\usetikzlibrary{trees}
\tikzstyle{every node}=[draw=black,thick,anchor=west]
\tikzstyle{selected}=[draw=blue]
\tikzstyle{optional}=[dashed,fill=gray!50]

\renewcommand{\dateseparator}{.}

\makeatletter
\newcommand*{\bdiv}{%
  \nonscript\mskip-\medmuskip\mkern5mu%
  \mathbin{\operator@font div}\penalty900\mkern5mu%
  \nonscript\mskip-\medmuskip
}
\makeatother

% Newer LaTeX versions should not add ligatures to listings, but for some reason
% it is not the case for me. As a result select PDF viewers copy wrong data.
\lstdefinestyle{ocbash}{
  language=bash,
  frame=tb,
  columns=fullflexible,
  captionpos=b,
  basicstyle=\ttfamily\normalsize,
  keepspaces=true,
  morekeywords={git, make, build, ioreg, grep, nvram, sort, sudo, diskutil, gfxutil, strings, dd, cut, python},
  literate =
    {"}{{\textquotedbl}}1
    {'}{{\textquotesingle}}1
    {-}{{-}}1
    {~}{{\texttildelow}}1
    {*}{{*}}1
    {fl}{{f{}l}}2
    {fi}{{f{}i}}2
    ,
}

\UseMicrotypeSet[protrusion]{basicmath} % disable protrusion for tt fonts
\PassOptionsToPackage{hyphens}{url} % url is loaded by hyperref

\makesavenoteenv{long table} % Fix footnotes in tables

% set default figure placement to htbp
\makeatletter
\def\fps@figure{htbp}
\makeatother

\providecommand{\tightlist}{%
  \setlength{\itemsep}{0pt}\setlength{\parskip}{0pt}}

\newcommand{\sectionbreak}{\clearpage}

\begin{document}

\begin{titlepage}
   \begin{center}
       \vspace*{2.0in}

       \Huge

       \IfFileExists{Logos/Logo.pdf}
         {\includegraphics[width=160pt, height=160pt]{Logos/Logo.pdf}}
         {\includegraphics[width=160pt, height=160pt]{../Logos/Logo.pdf}}

       \sffamily

       \textbf{OpenCore}

       \vspace{0.2in}

       Errata Sheet

       \vspace{0.2in}

        {[}\today{]}

       \normalsize

       \vfill

       \rmfamily

       Copyright \textcopyright 2020-2021 vit9696

   \end{center}
\end{titlepage}

\section{Introduction}\label{introduction}

This document provides information on select issues discovered in the relesed
versions of \href{https://github.com/acidanthera/OpenCorePkg}{OpenCore}
and provides workarounds whenever possible. This document is updated
after OpenCore is released and is thus not contained in the binary packages.

This document contains issues considered causing significant impact on the end-user.
More details on all known issues including feature requests and development version issues
can be found at \href{https://github.com/acidanthera/bugtracker}{Acidanthera Bugtracker}.

\section{Issue list}\label{issuelist}

\begin{itemize}
\item
  \textbf{Identifier}: \texttt{ERR059-1} \\
  \textbf{Published}: 2020-06-02 08:06 MSK \\
  \textbf{Updated}: 2020-06-21 18:22 MSK \\
  \textbf{Affected versions}: 0.5.9 \\
  \textbf{Resolved in}: 0.6.0 (\href{https://github.com/acidanthera/OpenCorePkg/commit/9d5aa267a9133fd9390b62012c21428fc77681f2}{9d5aa267}) \\
  \textbf{Description}:

  Boot picker menu may not show when upgrading macOS with the following error:

\texttt{OCUI: Entry kind 16 unsupported for Icon}\\
\texttt{OCB: ShowMenu failed - Unsupported}\\
\texttt{Halting on critical error}

  \textbf{Possible workarounds}:
  \begin{itemize}
    \tightlist
    \item Add \texttt{run-efi-updater} variable with \texttt{NO} string value \texttt{OR}
    \item Avoid using OpenCanopy \texttt{OR}
    \item Disable showing picker by setting \texttt{ShowPicker} to \texttt{false} \texttt{OR}
    \item Update to master version
  \end{itemize}


\item
  \textbf{Identifier}: \texttt{ERR059-2} \\
  \textbf{Published}: 2020-06-02 08:06 MSK \\
  \textbf{Updated}: 2020-06-21 18:22 MSK \\
  \textbf{Affected versions}: 0.5.9 \\
  \textbf{Resolved in}: 0.6.0 (\href{https://github.com/acidanthera/OpenCorePkg/commit/670d4e0c4f8538268367d16fc3ddef9b2ed13d46}{670d4e0c}) \\
  \textbf{Description}:

  Incorrect operating system may be selected as the default during macOS upgrade when
  Windows and macOS are installed on the same disk. Reference:
  \href{https://github.com/acidanthera/bugtracker/issues/948}{acidanthera/bugtracker\#948}.

  \textbf{Possible workarounds}:
  \begin{itemize}
    \tightlist
    \item Add \texttt{run-efi-updater} variable with \texttt{NO} string value \texttt{OR}
    \item Install Windows on a separate disk \texttt{OR}
    \item Update to master version
  \end{itemize}


\item
  \textbf{Identifier}: \texttt{ERR059-3} \\
  \textbf{Published}: 2020-06-21 18:22 MSK \\
  \textbf{Updated}: 2020-06-21 18:22 MSK \\
  \textbf{Affected versions}: 0.5.9 \\
  \textbf{Resolved in}: 0.6.0 (\href{https://github.com/acidanthera/OpenCorePkg/commit/2449e47cb2d110a288d491beee0b5b168d2bb480}{2449e47c}) \\
  \textbf{Description}:

  DEBUG builds may fail to load due to unaligned access assertion when accessing device paths.
  Reference: \href{https://github.com/acidanthera/bugtracker/issues/951}{acidanthera/bugtracker\#951}.

  \textbf{Possible workarounds}:
  \begin{itemize}
    \tightlist
    \item Avoid using DEBUG builds \texttt{OR}
    \item Update to master version
  \end{itemize}


\item
  \textbf{Identifier}: \texttt{ERR059-4} \\
  \textbf{Published}: 2020-06-24 21:18 MSK \\
  \textbf{Updated}: 2020-07-05 23:20 MSK \\
  \textbf{Affected versions}: 0.5.9 \\
  \textbf{Resolved in}: 0.6.0 (\href{https://github.com/acidanthera/OpenCorePkg/commit/d8ace476067ebb4b5abe17e4cdcd094fa3641689}{d8ace476}) \\
  \textbf{Description}:

  Driver injection into \texttt{kernel collections} normally used in macOS 11,
  and required for macOS 11 installer, is unsupported.

  \textbf{Possible workarounds}:
  \begin{itemize}
    \tightlist
    \item For \textbf{preview} version of KernelCollection injection code please update
    to commit \texttt{d8ace476} and make sure that variables forcing legacy
    kernel loading (\texttt{booter-fileset-kernel} and \texttt{booter-fileset-basesystem})
    are deleted from NVRAM.
    \item For \textbf{preinstalled} versions of macOS 11 add NVRAM variable \\
    \texttt{7C436110-AB2A-4BBB-A880-FE41995C9F82:booter-fileset-kernel}
    with \texttt{00} value; for some EfiBoot variants it may also be
    required to set \texttt{booter-fileset-basesystem} variable with \texttt{00} value.
  \end{itemize}


\item
  \textbf{Identifier}: \texttt{ERR059-5} \\
  \textbf{Published}: 2020-07-25 18:24 MSK \\
  \textbf{Updated}: 2020-07-25 18:24 MSK \\
  \textbf{Affected versions}: 0.0.3-0.5.9 \\
  \textbf{Resolved in}: 0.6.0 (\href{https://github.com/acidanthera/OpenCorePkg/commit/426a5dbe09365d3fe1d18dbde9868fe12900c30c}{426a5dbe}) \\
  \textbf{Description}:

  \texttt{OSXSAVE} CPUID feature may not be reported to macOS release kernels when emulating the CPUID
  causing crashes of select programs relying on \texttt{Hypervisor.framework} like Docker. Reference:
  \href{https://github.com/acidanthera/bugtracker/issues/1035}{acidanthera/bugtracker\#1035}.

  \textbf{Possible workarounds}:
  \begin{itemize}
    \tightlist
    \item Manually enable the \texttt{OSXSAVE} bit in \texttt{ECX} register \texttt{OR}
    \item Update to master version
  \end{itemize}

\item
  \textbf{Identifier}: \texttt{ERR060-1} \\
  \textbf{Published}: 2020-08-22 03:10 MSK \\
  \textbf{Updated}: 2020-08-22 03:10 MSK \\
  \textbf{Affected versions}: 0.0.1-0.6.0 \\
  \textbf{Resolved in}: 0.6.1 (\href{https://github.com/acidanthera/OpenCorePkg/commit/89be8ea5bb98dc8c9eb49a5e29c5343643535ae4}{89be8ea5}) \\
  \textbf{Description}:

  APFS-formatted macOS Recovery (e.g. Big Sur Recovery) will not load with a boot failure error. Reference:
  \href{https://github.com/acidanthera/bugtracker/issues/1078}{acidanthera/bugtracker\#1078}.

  \textbf{Possible workarounds}:
  \begin{itemize}
    \tightlist

    \item Enable \texttt{JumpstartHotPlug} \texttt{OR}
    \item Update to master version
  \end{itemize}

\item
  \textbf{Identifier}: \texttt{ERR061-1} \\
  \textbf{Published}: 2020-09-11 14:55 MSK \\
  \textbf{Updated}: 2020-09-11 14:55 MSK \\
  \textbf{Affected versions}: 0.5.3-0.6.1 \\
  \textbf{Resolved in}: 0.6.2 (\href{https://github.com/acidanthera/OpenCorePkg/commit/d3cf117c8661a4eda1863ca1fcdfae5c277974db}{d3cf117c}) \\
  \textbf{Description}:

  APFS-formatted drives previously made available with \texttt{ApfsImageLoader.efi}
  driver are no longer visible with OpenCore builtin \texttt{EnableJumpstart} APFS
  functionality on platforms that need \texttt{UnblockFsConnect} quirk. Reference:
  \href{https://github.com/acidanthera/bugtracker/issues/1128}{acidanthera/bugtracker\#1128}.

  \textbf{Possible workarounds}:
  \begin{itemize}
    \tightlist

    \item Update to master version \texttt{OR}
    \item Disable \texttt{ConnectDrivers} and enable \texttt{GlobalConnect} (may cause
      performance and compatibility issues depending on the driver use)

  \end{itemize}

\item
  \textbf{Identifier}: \texttt{ERR062-1} \\
  \textbf{Published}: 2020-10-25 01:22 MSK \\
  \textbf{Updated}: 2020-11-01 13:28 MSK \\
  \textbf{Affected versions}: 0.6.1-0.6.2 \\
  \textbf{Resolved in}: 0.6.3 (\href{https://github.com/acidanthera/OpenCorePkg/commit/1b0041493d4693f9505aa6415d93079ea59f7ab0}{1b004149}) \\
  \textbf{Description}:

  Booting Big Sur beta 10 or newer will cause a kernel panic in \texttt{apfs.kext} with the
  \texttt{Rooting from the live fs of a sealed volume is not allowed on a RELEASE build}
  message. Reference:
  \href{https://github.com/acidanthera/bugtracker/issues/1235}{acidanthera/bugtracker\#1235}.

  \textbf{Possible workarounds}:
  \begin{itemize}
    \tightlist

    \item Update to master version \texttt{OR}
    \item Set \texttt{SecureBootModel} to \texttt{Disabled}

  \end{itemize}

  \emph{Note}: On virtual machines \texttt{ForceSecureBootScheme} will also need
  to be enabled if \texttt{SecureBootModel} is different from \texttt{x86legacy}
  or \texttt{Disabled}.

\end{itemize}

\end{document}
